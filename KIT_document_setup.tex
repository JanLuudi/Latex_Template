%% --------------
%% | Typography |
%% --------------

% T1 font encoding
\usepackage[T1]{fontenc}
\usepackage[utf8]{inputenc}


% serif type: Linux Libertine
% \RequirePackage{libertine}
% Linux Libertine in math mode
% \RequirePackage[libertine]{newtxmath}

%% --------------
%% | Packages   |
%% --------------
% \usepackage[nooneline,bf,justification=centerlast]{caption} %To use centered captions (Is this against Chicago Manual of style?)
\usepackage[nooneline,bf]{caption} % Figure descriptions from left margin
\usepackage{times}
% \usepackage{titling}

% numbers option provides compact numerical references in the text.
\usepackage[numbers]{natbib}

%% -------------------------------
%% | Graphics, Figures, Tables   |
%% -------------------------------
\usepackage[dvips,pdftex]{graphicx}
\usepackage{epic}
\usepackage{eepic}
\usepackage{epsfig}
\usepackage{tikz}
\usepackage{pgfplots}
\usepackage{float} % damit Tabellen und Grafiken sich selbst ausrichten dürfen und nicht zwangshaft an einer Stelle stehen. verhindern kann man das mit [H]
\usepackage{subcaption}
% \usepackage[table]{xcolor}
% \usepackage{color}
\usepackage{transparent}

\usepackage{colortbl}
\usepackage{multicol}
\usepackage{multirow}
\usepackage{longtable}

\usepackage{rotating}
\usepackage{hhline}

%% --------------
%% | Listings    |
%% --------------
\usepackage{algorithm}		  % Code-Listings
\usepackage{algpseudocode}    % algorithmicx with layout
%\usepackage{algorithmic}	  % Code-Listings
% see http://www.ctan.org/tex-archive/macros/latex/contrib/algorithm2e/algorithm2e.pdf
% for more sophisticated algorithm listings
\usepackage{listings}

\lstset{
language=C++,
basicstyle=\small\ttfamily,
numbers=left,
numbersep=5pt,
xleftmargin=20pt,
frame=tb,
framexleftmargin=20pt
}
% Make listings similar to algorithm environment
\DeclareCaptionFormat{mylst}{\hrule#1#2#3}
\captionsetup[lstlisting]{format=mylst,labelfont=bf,singlelinecheck=off,labelsep=space}


%% ------------------------------------------
%% | Glossary, Acronyms and Nomenclature    |
%% ------------------------------------------
\usepackage[acronym,toc]{glossaries}
\usepackage{nomencl}

\makeglossaries

%% --------------
%% | Margins    |
%% --------------
\usepackage{vmargin}          % Adjust margins in a simple way
\usepackage[absolute,overlay]{textpos}

%% --------------
%% | Math       |
%% --------------
\usepackage{grffile}
\usepackage{amsmath}
\usepackage{amstext}
\usepackage{amssymb}
\usepackage{bm}
% https://www.dpg-physik.de/dpg/gliederung/junge/rg/wuerzburg/Archiv/WS%202011-12/LaTeX/siunitx.pdf
\usepackage{siunitx} % für alle SI-Einheiten (siehe Siunitx.pdf) \SI[per-mode=symbol]{56}{\gram\per\square\meter}

\usepackage{textcomp}
\usepackage{enumerate}
\usepackage[inline]{enumitem}


%% ------------------------------
%% | TODOs and Dummy Text       |
%% ------------------------------
\ifKITthesis@final
  \usepackage[disable]{todonotes}
\else
\iflanguage{ngerman}{
  \usepackage[colorinlistoftodos,ngerman]{todonotes} %See: https://www.ra.informatik.tu-darmstadt.de/fileadmin/user_upload/Group_RA/eiwa/todonotes.pdf
}{
  \usepackage[colorinlistoftodos]{todonotes} %See: https://www.ra.informatik.tu-darmstadt.de/fileadmin/user_upload/Group_RA/eiwa/todonotes.pdf
}
\fi

\usepackage{blindtext}

%% ---------------------------
%% | References in PDF       |
%% ---------------------------
% Hyperref should be the last package loaded
\usepackage[hyphens]{url}
\usepackage[breaklinks,colorlinks=true,pdftex,pageanchor=false]{hyperref}


% My Todo commands
\newcommand{\todoin}[1]{\todo[inline]{#1}}
\newcommand{\todoBetter}[1]{\todo[color=blue]{#1}}
\newcommand{\todoBetterin}[1]{\todo[color=blue,inline]{#1}}
\newcommand{\todoAdd}[1]{\todo[color=green]{#1}}
\newcommand{\todoAddin}[1]{\todo[color=green,inline]{#1}}
\newcommand{\todoCopied}[1]{\todo[color=olive]{This is copied and needs to be reworded: #1}}
\newcommand{\todoCopiedin}[1]{\todo[color=olive,inline]{This is copied and needs to be reworded:} {\color{olive}#1}}
\newcommand{\todoImportant}[1]{\todo[color=red]{IMPORTANT: #1}}
\newcommand{\todoImportantin}[1]{\todo[color=red,inline]{IMPORTANT: #1}}
\newcommand{\todoCheckThis}[1]{\todo[color=pink]{CHECK THIS: #1}}
\newcommand{\todoCheckThisin}[1]{\todo[color=pink,inline]{CHECK THIS: #1}}


\newcommand{\kithref}[2]{\href{#1}{\textit{#2}}}
% \newcommand{\see}[1]{cf.

%% --- End of Packages ---


%% -------------------------------
%% |        Declarations         |
%% -------------------------------
\pgfplotsset{compat=newest}
\pgfplotsset{plot coordinates/math parser=false}
%% the following commands are needed for some matlab2tikz features
\usetikzlibrary{plotmarks}
\usetikzlibrary{arrows.meta}
\usepgfplotslibrary{patchplots}

\graphicspath{{Figures/}{../jpeg/png/svg/gif/}}

\newcommand\setpdf{

  \let\theauthor\@author
  \let\thetitle\@title

  \hypersetup{
%     bookmarks=true,         % show bookmarks bar?
%     unicode=true,          % non-Latin characters in Acrobat's bookmarks
%     pdftoolbar=true,        % show Acrobat's toolbar?
%     pdfmenubar=true,        % show Acrobat's menu?
%     hyperindex=true,
    plainpages=false,
%     pdfpagelabels=true,
%     pdffitwindow=false,     % window fit to page when opened
%     pdfstartview={FitH},    % fits the width of the page to the window
%     pdftitle={\thethesistype},    % title
%     pdfauthor={\@author},     % author
%     pdfsubject={\thetitle},   % subject of the document
%     pdfcreator={\@author},   % creator of the document
%     pdfproducer={\@author}, % producer of the document
%     pdfkeywords={\keywords}, % list of keywords
%     pagebackref =true,
%     pdfnewwindow=true,      % links in new window
%     colorlinks=true,       % false: boxed links; true: colored links
    linkcolor=black,          % color of internal links (change box color with linkbordercolor)
    citecolor=black,        % color of links to bibliography
    filecolor=black,      % color of file links
    urlcolor=black           % color of external links
%     pdfborder={0 0 0},%
%     %linkcolor=kit-blue100,%
%     %citecolor=kit-green100,%
%     %urlcolor=kit-red100
  }
}

\newcommand\tab[1][1cm]{\hspace*{#1}}

\newcommand{\comment}[1]{}

\ifKITthesis@final
    \newcommand{\afterReview}[1]{#1}
    \newcommand{\afterReviewFig}[1]{#1}
\else
    \newcommand{\afterReview}[1]{{\color{magenta}#1}}
    \newcommand{\afterReviewFig}[1]{\colorbox{magenta}{#1}}
\fi

%% --- End of Declarations ---

