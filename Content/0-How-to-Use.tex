%% ==============================
\chapter{How to use this Template}
\label{sec:how-to-use}
%% ==============================

\textbf{IMPORTANT:} This chapter will disappear when you add final parameter on the document. See section \ref{ssec:finalizing_the_document}.

\subsection{Getting Started}
Initially you \textbf{should only edit} the \texttt{My\_document\_info.tex} with important data regarding your work.

Add \textbf{content} in files in \texttt{Content} folder.

Add \textbf{bibliography} in the file \texttt{Bibliography/my\_thesis\_bibliography.bib} or just add a file from your supervisior in the \texttt{Bibliography} folder and reference it in the \texttt{\textbackslash mybibliographyfiles} command in the \texttt{My\_document\_info.tex} file.

As an useful aid in all scientific work following book is recommended: \cite{deininger2005studien}.


\subsection{Inline lists}
My robot can:
\begin{enumerate*}[label=(\roman*)]
 \item forward and backward movements,
 \item sidewards movements,
 \item rotation along any curve in space,
 \item place of artificial forces along paths.
\end{enumerate*}

\begin{enumerate*}[label=(\arabic*),itemjoin={{; }}]
    \item the independently controllable wheels
    \item the rechargeable battery pack
    \item the Sick LMS100 laser range scanner
    \item the force-torque sensor
    \item the handlebar for controlling the robotic device
\end{enumerate*}

\url{https://ctan.math.illinois.edu/macros/latex/contrib/enumitem/enumitem.pdf}

\subsection{Todos}

Todo command can be used in multiple form and paramters set. You can set todos on the right side with commands:
{\small
\begin{verbatim}
\todo{Rewrite this section}
\todo[color=green]{Stuff}
\end{verbatim}
} which render as:
\todo{Rewrite this section}
\todo[color=green]{Stuff}



% \todo[due=2017-08-18]{Stuff}
% \todo[done]{Stuff}

You can also create inline todos with command:
{\small
\begin{verbatim}
\todo[inline]{Rewrite this section}
\todo[inline,color=green]{Rewrite this section}
\todoin{Stuff}
\end{verbatim}
} which rendrs as:
\todo[inline]{Rewrite this section}
\todo[inline,color=green]{Rewrite this section}
\todoin{Stuff}

One can also use command for figure placeholder with command:
{\small
\begin{verbatim}
 \missingfigure{Please add some figures}
\end{verbatim}
} which renders as:
\missingfigure{Please add some figures}


\subsubsection{User defined version of ToDos for easier usage}

\paragraph{Short for inline}
\begin{verbatim}
 \todoin{This I have to do}
\end{verbatim}
\todoin{This I have to do}

\paragraph{Remember to rewrite something better}
\begin{verbatim}
 \todoBetter{This I have to do}
\end{verbatim}
\todoBetter{This I have to do}
\begin{verbatim}
 \todoBetterin{This I have to do}
\end{verbatim}
\todoBetterin{This I have to do}

\paragraph{Remember to add some stuff later}
\begin{verbatim}
 \todoAdd{This I have to do}
\end{verbatim}
\todoAdd{This I have to do}
\begin{verbatim}
 \todoAddin{This I have to do}
\end{verbatim}
\todoAddin{This I have to do}

\paragraph{Remember to remove some copy-pasted text}
\begin{verbatim}
 \todoCopied{This I have to do}
\end{verbatim}
\todoCopied{This I have to do}
\begin{verbatim}
 \todoCopiedin{This I have to do}
\end{verbatim}
\todoCopiedin{This I have to do}

\paragraph{Remember that something is important to consider in the future}
\begin{verbatim}
 \todoImportant{This I have to do}
\end{verbatim}
\todoImportant{This I have to do}
\begin{verbatim}
 \todoImportantin{This I have to do}
\end{verbatim}
\todoImportantin{This I have to do}

\paragraph{Remember that something need to be checked}
\begin{verbatim}
 \todoCheckThis{This I have to do}
\end{verbatim}
\todoCheckThis{This I have to do}
\begin{verbatim}
 \todoCheckThisin{This I have to do}
\end{verbatim}
\todoCheckThisin{This I have to do}


\subsection{Glossaries and Acronyms}
Please use \texttt{glossaries} package for this. See \href{https://en.wikibooks.org/wiki/LaTeX/Glossary}{\textit{documentation}}.

Example (Acronym):
{\small
\begin{verbatim}
\newacronym{ipr}{IAR-IPR}{Institute for Anthropomatics and Robotics - Intelligent Process Control and Robotics}
\end{verbatim}
} \newacronym{ipr}{IAR-IPR}{Institute for Anthropomatics and Robotics - Intelligent Process Control and Robotics}
is used by
{\small
\begin{verbatim}
\gls{ipr}
\end{verbatim}
}
rendering as ``\gls{ipr}'', on the first use and as ``\gls{ipr}'' on every following use. For further feature see \href{https://en.wikibooks.org/wiki/LaTeX/Glossary}{\textit{documentation}}.

Please keep in mind that one has to call \href{https://www.dickimaw-books.com/cgi-bin/faq.cgi?action=view&categorylabel=glossaries#noglossary}{\textit{external commands}} for glossaries to work.


\subsection{Nomenclature}
For more details see \href{https://tex.stackexchange.com/questions/27824/using-package-nomencl}{\textit{example}}.

Use following command:
\textbackslash nomenclature\{IAR-IPR\}\{Institute for Anthropomatics and Robotics (IAR) - Intelligent Process Control and Robotics (IPR)\}
\nomenclature{IAR-IPR}{Institute for Anthropomatics and Robotics (IAR) - Intelligent Process Control and Robotics (IPR)}


\subsection{SI Units}
Please use \texttt{siunitx} package for this. See:  \url{https://ctan.org/pkg/siunitx}

\subsection{Tables}
\begin{table}[H]
\caption{Tables have caption on top.}
\label{tab:table_caption}
\centering
\resizebox{\columnwidth}{!}{
 \begin{tabular}{| c | c | c | c |}
  \hline
  Object & Speed $[cm/s]$ & Inner LR $[cm]$ & Inner UR $[cm]$ \\ \hline \hline
  \multirow{3}{*}{\emph{Pitcher}} & real & $ n/a $ & $ 5.65 $ \\
   & $4.60$ & $3.71 \pm 0.67$ & $5.09 \pm 2.23$ \\
   & $10.64$ & $3.55 \pm 0.57$ & $6.14 \pm 0.69$ \\ \hline \hline
  \multirow{3}{*}{Cookie O} & real & $ 7.55 $ & $ 7.55 $ \\
   & $4.60$ & $6.98 \pm 0.27$ & $6.98 \pm 0.27$ \\
   & $10.64$ & $6.77 \pm 0.26$ & $6.77 \pm 0.26$ \\ \hline
 \end{tabular}
 }
\end{table}

Use \texttt{\textbackslash longtable} for tables over multiple pages. See \href{https://de.wikibooks.org/wiki/LaTeX-W%C3%B6rterbuch:_longtable_(Umgebung)}{documentation}.

\subsection{Figures}
\begin{figure}[H]
    \centering
    \includegraphics[width=0.8\columnwidth]{Logos/KIT-Departments/IPRLogo_en}
    \caption{Figures have caption under. If you use figures from other work, do not forget to reference them \cite{deininger2005studien}.}
    \label{fig:figure_caption}
\end{figure}

\subsection{Subfigures}
%For more explanation see \href{https://en.wikibooks.org/wiki/LaTeX/Floats,_Figures_and_Captions#Subfloats}{Wikibooks}
\begin{figure}[H]
    \centering
    \begin{subfigure}[b]{0.3\textwidth}
        \includegraphics[width=\textwidth]{Logos/KIT-Departments/IPRLogo_en}
        \caption{First logo}
        \label{fig:logo1}
    \end{subfigure}
    ~ %add desired spacing between images, e. g. ~, \quad, \qquad, \hfill etc.
      %(or a blank line to force the subfigure onto a new line)
    \begin{subfigure}[b]{0.3\textwidth}
        \includegraphics[width=\textwidth]{Logos/KIT-Departments/IPRLogo_en}
        \caption{Second logo}
        \label{fig:logo2}
    \end{subfigure}
    ~ %add desired spacing between images, e. g. ~, \quad, \qquad, \hfill etc.
    %(or a blank line to force the subfigure onto a new line)
    \begin{subfigure}[b]{0.3\textwidth}
        \includegraphics[width=\textwidth]{Logos/KIT-Departments/IPRLogo_en}
        \caption{Third logo}
        \label{fig:logo3}
    \end{subfigure}
    \caption{Pictures of Logos}\label{fig:logos}
\end{figure}


\subsection{Citation}

\todoin{Add citet und citep}

\subsubsection{Multiple citations}
Use multiple citation like this:
{\small
\begin{verbatim}
\cite{deininger2005studien, deininger2005studien}
\end{verbatim}
}
rendered as ``\cite{deininger2005studien, deininger2005studien}''.


\subsubsection{More powerfull cite commands: \texttt{\textbackslash citet} and \texttt{\textbackslash citep}}
For comprehensive description please check \href{http://merkel.texture.rocks/Latex/natbib.php}{\textit{the natbib documentation}}.

Rather than using the awkward construction\footnote{The example is from the template for the conference \href{http://www.roboticsconference.org/information/authorinfo/}{\textit{Robotic Science and Systems}}.}
{\small
\begin{verbatim}
\cite{deininger2005studien} describes...
\end{verbatim}
}
\noindent
rendered as ``\cite{deininger2005studien} demonstrated...,'' or the inconvenient
{\small
\begin{verbatim}
Deininger \cite{deininger2005studien} describes...
\end{verbatim}
}
\noindent
rendered as ``Deininger \cite{deininger2005studien} demonstrated...'', one can write
{\small
\begin{verbatim}
\citet{deininger2005studien} describes...
\end{verbatim}
}
\noindent
which renders as ``\citet{deininger2005studien} demonstrated...'' and is
both easy to write and much easier to read.


\textbf{Citing specific chapter}:

\citet[sec. III]{Kroger2008MultiSensor}

\citep[sec. III]{Kroger2008MultiSensor}

For more examples check \href{http://merkel.texture.rocks/Latex/natbib.php}{\textit{the natbib documentation}}.

\subsection{Using Hyperlinks}
Please use the ability of PDF viewers to interpret hyperlinks\footnote{The example is from the template for the conference \href{http://www.roboticsconference.org/information/authorinfo/}{\textit{Robotic Science and Systems}}.}, specifically to allow each reference in the bibliography to be a
link to an online version of the reference.
As an example, if you were to cite ``Passive Dynamic Walking''
\cite{McGeer01041990}, the entry in the bibtex would read:

{\tiny
\begin{verbatim}
@article{McGeer01041990,
  author = {McGeer, Tad},
  title = {\href{http://ijr.sagepub.com/content/9/2/62.abstract}{Passive Dynamic Walking}},
  volume = {9},
  number = {2},
  pages = {62-82},
  year = {1990},
  doi = {10.1177/027836499000900206},
  URL = {http://ijr.sagepub.com/content/9/2/62.abstract},
  eprint = {http://ijr.sagepub.com/content/9/2/62.full.pdf+html},
  journal = {The International Journal of Robotics Research}
}
\end{verbatim}
}
\noindent
and the entry in the compiled PDF would look like:

\def\tmplabel#1{[#1]}

\begin{enumerate}
\item[\tmplabel{1}] Tad McGeer. \href{http://ijr.sagepub.com/content/9/2/62.abstract}{Passive Dynamic
Walking}. {\em The International Journal of Robotics Research}, 9(2):62--82,
1990.
\end{enumerate}
%
where the title of the article is a link that takes you to the article on IJRR's website.


Also use this for adding links into text as done in the \footnotemark[2]. For more information see documentation on \href{https://de.wikibooks.org/wiki/LaTeX-W%C3%B6rterbuch:_hyperref}{wikibooks}. The \texttt{hyperref} package is already configured for this document in \texttt{KIT\_document\_setup.tex} file.


\subsection{Equations}
Use numbered equations:
\begin{equation} \label{equ:equ}
  m \cdot \ddot{x}(t) + d \cdot \dot{x}(t) = F(t)
\end{equation}


\subsection{Inline comments}
Use command \texttt{\textbackslash comment\{\}} for inline comments.

\subsection{After Review marking}
Use command \texttt{\textbackslash afterReview\{\}} for marking text parts as \afterReview{changed}.

\subsection{Finalizing the Document}
\label{ssec:finalizing_the_document}
Please check here: \url{https://github.com/KITrobotics/Latex_Template/blob/master/README.md#finalizing-document}
