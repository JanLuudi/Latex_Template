%% ==============================
\chapter{\iflanguage{ngerman}{Stand der Wissenschaft und Technik}{State of the art}}
\label{sec:state_of_the_art}
%% ==============================
\setlength{\parindent}{0pt}

\section{Vorgaben zu Wiederverwendungs- und Recyclingquoten}
\label{sec:VorgabenWRQ}

Im Jahr 2018 wurden in Deutschland rund 362 Millionen Tonnen Abfälle erfasst \cite{DESTATIS2018}. Das sind etwa drei Millionen Tonnen Abfälle mehr als im Vorjahr und führt den seit 2012 steigenden Trend fort. Das Gesamtabfallaufkommen setzt sich zusammen aus Bau- und Abbruchabfällen, Siedlungsabfällen, gefährlichen Abfällen und Abfällen aus Produktion und Gewerbe.

Wie in \citet{Bilitewski} beschrieben, ist in den vergangenen 50 Jahren eines der Ziele der Abfallwirtschaft in den Vordergrund gerückt, nämlich die Entsorgungswege so zu gestalten, dass das Wohl der Allgemeinheit nicht beeinträchtigt wird.  
\begin{quote}
	Unter dem Begriff Abfallwirtschaft versteht man
	\textit{" [...] die Summe aller Maßnahmen zur Vermeidung, möglichst schadlosen Behandlung, Wieder- und Weiterverwendung, Verwertung als Ressource und endgültigen Unterbringung von Abfällen aller Art unter der Berücksichtigung ökologischer und ökonomischer Gesichtspunkte"} \cite{Bilitewski}.
\end{quote}

Für die Entsorgung und die Wiederverwendung von Abfällen gelten strenge Richtlinien und Anforderungen, wie die Europäische Abfallrahmenrichtlinie und das Kreislaufwirtschaftsgesetz auf Bundesebene. Die diesbezüglich in Art. 4 Abs. 1 lit. a-e Europäische Abfallrahmenrichtlinie und in §6 KrWG festgelegte Abfallhierarchie ist eine Prioritätenfolge und dient sowohl der Abfallvermeidung, als auch der Abfallbewirtschaftung \cite{EURichtlinie2008}. Am wichtigsten ist gemäß dieser Abfallhierarchie die Vermeidung, danach folgt die Vorbereitung zur Wiederverwendung, das Recycling, sonstige Verwertungsmaßnahmen und als letzte Stufe die Beseitigung.

Die Begriffe Verwertung, Wiederverwendung und Recycling müssen für ein besseres Verständnis voneinander abgegrenzt werden. Alle drei Begriffe sind in Abfallrahmenrichtlinie definiert.

Die Verwertung umfasst demnach \glqq jedes Verfahren, als dessen Hauptergebnis Abfälle innerhalb der Anlage oder in der weiteren Wirtschaft einem sinnvollen Zweck zugeführt werden, indem sie andere Materialien ersetzen, die ansonsten zur Erfüllung einer bestimmte Funktion verwendet worden wären, oder die Abfälle so vorbereitet werden, dass sie diese Funktion erfüllen.\grqq \cite{EURichtlinie2008}.
\\
Unterteilt ist die Verwertung in drei Arten. Die rohstoffliche Verwertung, die werkstoffliche Verwertung und die energetische Verwertung \cite{Bilitewski}.

Somit schließt die Verwertung Wiederverwendungs- und Recyclingverfahren, die zur roh- und werkstofflichen Verwertung gehören, mit ein. Zu beachten ist aber, dass auch die Nutzung der Abfälle als Ersatzbrennstoff oder als anderes Mittel der Energieerzeugung in bestimmten Verbrennungsanlagen als Verwertung gilt.

Als Wiederverwendung bezeichnet man \glqq jedes Verfahren, bei dem Erzeugnisse oder Bestandteile, die keine Abfälle sind, wieder für denselben Zweck verwendet werden, für den sie ursprünglich bestimmt waren.\grqq \cite{EURichtlinie2008}. Auch die Vorbereitung zur Wiederverwendung ist eine Art der Abfallvermeidung und wird in den Wiederverwendungs- und Recyclingquoten berücksichtigt.

Recycling umfasst "jedes Verwertungsverfahren, durch das Abfallmaterialien zu Erzeugnissen, Materialien oder Stoffen entweder für den ursprünglichen Zweck oder für andere Zwecke aufbereitet werden. Es schließt die Aufbereitung organischer Materialien ein, aber nicht die energetische Verwertung und die Aufbereitung zu Materialien, die für die Verwendung als Brennstoff oder zur Verfüllung bestimmt sind" \cite{EURichtlinie2008}. Diese Verfahren finden gerade bei Stoffen wie Papier, Pappe, Kunststoff, Glas, Baustoffen, Metallen und Holz Anwendung.

Der in der Abfallrahmenrichtlinie für 2020 festgelegte Mindestanteil für die Vorbereitung zur Wiederverwendung und dem Recycling von Siedlungsabfällen liegt bei 50 Gewichtsprozent. Die Vorgabe für Bau- und Abbruchabfälle, welche in Europa einen Anteil von circa 32 Gewichtsprozent des gesamten Abfallaufkommens ausmachen, liegt bei 70 Gewichtsprozent.
Des Weiteren existieren europäische Vorgaben zur Wiederverwendung und dem Recycling von Siedlungsabfällen. Bis zum Jahr 2025 sollen 55 Gewichtsprozent, bis 2030 60 Gewichtsprozent und bis 2035 65 Gewichtsprozent der gesamten Siedlungsabfälle der Kreislaufwirtschaft zugeführt werden \cite{EURichtlinie2008}. Diese beiden Vorgaben wurden aus der EU-Verordnung in das deutsche Kreislaufwirtschaftsgesetz übernommen.

Die tatsächliche Wiederwerwendungs- und Recyclingquote von Siedlungsabfällen lag 2018 in Deutschland bei 68 Gewichtsprozent \cite{DESTATIS2018}, was im Europäischen Vergleich der höchste Wert ist \cite{EUWaste}.
Auch in Bezug auf die Wiederwerwendungs- und Recyclingquote von nicht gefährlichen Bau- und Abbruchabfällen, welche 2018 in Deutschland bei 89 Gewichtsprozent lag \cite{DESTATIS2018}, werden die europäischen Vorgaben eingehalten.

Ein rohstoffreicher Teil der Siedlungsabfälle sind Verpackungen. Für die verschiedenen Fraktionen gelten individuelle Recyclingquoten, die in EU-Richtlinie 94/62/EG, sowie im §16 Verpackungsgesetz \cite{VerpackG2017} geregelt sind (siehe \ref{tab:Quoten1}). Im Jahr 2017 lag der Recyclinganteil der gesamten Kunststofffraktion bei 48,7\% \cite{BMU}. 

\begin{table}[]
	\begin{tabular}{|l|l|l|}
		\hline
		& \textbf{bisher} & \textbf{bis 2022} \\ \hline
		Glas                         & 80\%            & 90\%              \\ \hline
		Papier, Pappe, Karton        & 85\%            & 90\%              \\ \hline
		Eisenmetalle                 & 80\%            & 90\%              \\ \hline
		Aluminium                    & 80\%            & 90\%              \\ \hline
		Kunststoffe                  & 65\%            & 70\%              \\ \hline
		Getränkekartonverpackungen   & 75\%            & 80\%              \\ \hline
		sonstige Verbundverpackungen & 55\%            & 70\%              \\ \hline
	\end{tabular}
	\centering
	\caption{Vorgaben zu Wiederverwendungs- und Recyclingquoten aus §16 Verpackungsgesetz}
	\label{tab:Quoten1}
\end{table}

In der EU-Richtlinie wird die Vorgabe gemacht, bis 2025 mindestens 65 Gewichtsprozent der Verpackungsabfälle zu recyceln und dabei die folgenden Quoten zu erreichen \ref{tab:Quoten2}.


\begin{table}[]
	\begin{tabular}{|l|l|l|}
		\hline
		& \textbf{bis 2025} & \textbf{bis 2030} \\ \hline
		Verpackungsabfälle gesamt & 65\%              & 70\%              \\ \hline
		Glas                      & 70\%              & 75\%              \\ \hline
		Papier, Pappe, Karton     & 75\%              & 85\%              \\ \hline
		Eisenmetalle              & 70\%              & 80\%              \\ \hline
		Aluminium                 & 50\%              & 60\%              \\ \hline
		Kunststoffe               & 50\%              & 55\%              \\ \hline
	\end{tabular}
	\centering
	\caption{Vorgaben zu Wiederverwendungs- und Recyclingquoten aus EU-Richtlinie 94/62/EG \cite{EuropaischeKommission2018}}
	\label{tab:Quoten2}
\end{table}


Durch diese Vorgaben wird die Recyclingbranche vor neue Aufgaben gestellt und es müssen neue Technologien gefunden und bestehende Systeme verbessert werden. Ein Beispiel für eine etablierte Technologie in diesem Bereich ist die automatische optische Sortierung.
Wie in \citet{Bilitewski} beschrieben, findet die optische Sortiertechnik in Sortieranlagen für Abfälle seit über 20 Jahren Anwendung und wird für die Sortierung von Farbglas, Verpackungs- und technischen Kunststoffen, Verbundverpackungen und Papier genutzt.

\clearpage
\section{Statistische Versuchsplanung}

Für die Planung von Versuchen lässt sich auf eine Vielzahl von Vorgehensweisen zurückgreifen. Das grundsätzliche Ziel ist jedoch meistens dasselbe, nämlich den Einfluss verschiedener Faktoren auf die relevanten Zielgrößen zu ermitteln.
Vorgehen wie die Ein-Faktor-Methode sind allerdings in den meisten Fällen nicht ausreichend aufschlussreich. Bei dieser Methode wird immer nur ein Einflussfaktor variiert, während die anderen Faktoren konstant gehalten werden. Dies führt einerseits zu einer sehr hohen Anzahl an Versuchen, aber auch dazu, dass bei  Wechselwirkungen unter den Einflussfaktoren falsche Schlussfolgerungen gezogen werden und ist daher nicht für jedes Szenario geeignet.

Ein Vorgehen, was nicht nur solche Wechselwirkungen erkennbar machen kann, sondern auch den Versuchsumfang überschaubar hält ist die sogenannte Statistische Versuchsplanung.
Hierbei handelt es sich um ein standardisiertes Vorgehen in der Versuchsplanung und -auswertung. 
Die Methode unterscheidet sich dadurch vom Ein-Faktor-Verfahren, dass  mehrere Faktoren gleichzeitig variiert werden. Dies sorgt dafür, dass nicht nur die einzelnen Effekte der Faktoren erkannt werden, sondern auch die Wechselwirkungen dieser untereinander.

In dieser Arbeit wird die statistische Versuchsplanung für die Planung und Auswertung der Technikumsversuche verwendet. Hierdurch wird die Vorgehensweise erläutert und kann auch für Versuche im naturgetreuen Umfang eine Hilfestellung bieten. Gerade in Bezug auf die in \ref{sec:VorgabenWRQ} beschriebenen wachsenden Anforderungen an die Recyclingbranche, sorgen dafür, dass neue Trennverfahren erprobt, oder bestehende Verfahren verbessert werden müssen \todoin{hier Paper Vorkonditionierung Küppers und DOE Shredder zitieren}. Um Versuche so realistisch wie möglich zu gestalten, reicht der Technikumsumfang meist nicht aus. Bei der Durchführung von Versuchen im naturgetreuen Umfang entstehen hohe Kosten für die Versuche, welche es aus wirtschaftlichem Aspekt zu minimieren gilt.


\todoin{Vorgehensweise beschreiben}

Wie in \citet{Siebertz2016} beschrieben, ist die Vorgehensweise bei der statistischen Versuchsplanung weltweit standardisiert.
Zunächst muss das zu untersuchende System abgegrenzt werden. Dafür werden die Parameter, Faktoren und Qualitätsmerkmale identifiziert. Als Parameter bezeichnet man laut \cite{Siebertz2016} alle Eingangsgrößen. Aus diesen wählt man anschließend diejenigen aus, deren Einfluss auf die Qualitätsmerkmale untersucht werden soll. Die in dieser Teilmenge vorhandenen Parameter werden Faktoren genannt.
Ergebnisse eines Systems - zum Beispiel die Sortierqualität einer Sortiermaschine - lassen sich in den meisten Fällen quantifizieren. Diese "messbaren Ergebnisse" \cite{Siebertz2016} werden Qualitätsmerkmale genannt.

Die Einstellungen, welche für die verschiedenen Faktoren in den Versuchen verwendet werden, werden im Vorhinein auf verschiedene Stufen festgelegt \cite{Siebertz2016}. Es ist hierbei wichtig, geeignete Stufenabstände zu wählen, da der tatsächliche Einfluss eines Faktoren bei zu kleinen Abstand nicht vom Messrauschen zu unterscheiden ist.



Nachdem die Faktoren und Qualitätsmerkmale festgelegt wurden, muss ein passender Versuchsplan gefunden werden. Wie schon erwähnt zeichnet sich die statistische Versuchsplanung dadurch aus, dass sie den Versuchsaufwand überschaubar hält und trotzdem gute Ergebnisse liefert. Grund dafür ist der orthogonale Aufbau der Versuchspläne. Laut \citet{Siebertz2016} spricht man von einem orthogonalen Versuchsplan dann, \glqq wenn keine Kombination aus jeweils zwei Spalten miteinander korreliert.\grqq
Dadurch hängt keines der Einstellungsmuster für einen Faktor von einem Anderen ab, wodurch diese unabhängig voneinander sind. Somit kann der jeweilige Effekt, den Faktoren auf die Qualitätsmerkmale haben, berechnet werden. Auch die Wechselwirkungen unter mehreren Faktoren können so ermittelt werden. Am anschaulichsten wird dies an einem vollfaktoriellen Versuchsplan (siehe \ref{tab:Vollfaktorplan}).

\begin{table}
	\centering
	\begin{tabular}{c|ccccccc}
		& F1 & F2 & F3 & F1xF2 & F1xF3 & F2xF3 & F1xF2xF3  \\ 
		\hline
		1 & -  & -  & -  & +     & +     & +     & -         \\
		2 & +  & -  & -  & -     & -     & +     & +         \\
		3 & -  & +  & -  & -     & +     & -     & +         \\
		4 & +  & +  & -  & +     & -     & -     & -         \\
		5 & -  & -  & +  & +     & -     & -     & +         \\
		6 & +  & -  & +  & -     & +     & -     & -         \\
		7 & -  & +  & +  & -     & -     & +     & -         \\
		8 & +  & +  & +  & +     & +     & +     & +        
	\end{tabular}
	\caption{Vollfaktorieller Versuchsplan für 3 Faktoren auf 2 Stufen mit allen Kombinationen}
	\label{tab:Vollfaktorplan}
\end{table}


Dass die oben genannten Vorteile der statistischen Versuchsplanung in fast allen Anwendungsfällen gelten liegt laut \citet{Siebertz2016} an den verschiedenen Versuchsplandesigns für unterschiedliche Problemstellungen. Die Autoren sprechen hier von Screening Versuchsplänen, Versuchsplänen für ein quadratisches Beschreibungsmodell, Mischungspläne, und maßgeschneiderten Versuchsplänen.


Wie in \citet{Siebertz2016} beschrieben, sind Screening Versuchspläne dann geeignet, wenn eine hohe Zahl an Faktoren vorhanden ist. Versuchspläne für ein quadratisches Beschreibungsmodell eignen sich dann, wenn nichtlineare Zusammenhänge zwischen Faktor und Qualitätsmerkmal erwartet werden. Hier reicht ein zweistufiger Versuchsplan nicht aus und es muss mindestens eine mittlere Stufe festgelegt werden.
Die Autoren in \citet{Siebertz2016} nennen diesbezüglich das Central-Composite-Design und das Face-Centered-Central-Composite-Design, welche auf einem zweistufigen Versuchsplan aufbauen und durch zusätzliche Versuche nichtlineare Zusammenhänge erkennbar machen. Das Vorgehen der Versuchsplanung und -Durchführung mit dem Face-Centered-Central-Composite-Design wird in \citet{Qin2009} beschrieben.

\todoin{CCD und FCCCD erläutern}

Weitere Designs für quadratische Beschreibungsmodelle sind das Box-Behnken-Design und das Monte-Carlo-Verfahren.

Auch Beschreibungsmodelle höherer Ordnung können mithilfe der statistischen Versuchsplanung erstellt werden. Dies geht aber mit umfangreicheren Versuchsplänen einher und kann zu sogenanntem Overfitting des Modells an die Versuchsdaten führen \cite{Siebertz2016}. Anomalien, welche bei der Versuchsdurchführung einmalig auftreten, könnten dadurch ein sehr feines Modell verfälschen.
Daher wird in \citet{Siebertz2016} auch darauf hingewiesen, dass die Nichtlinearität in den meisten Fällen überschätzt wird und somit zweistufige Versuchspläne bei passend gewählten Stufenabständen zu guten Ergebnissen führen, um die Zusammenhänge im System zu ermitteln.


In der Verfahrenstechnik und der Chemie werden oft Mischungspläne eingesetzt \cite{Siebertz2016}. Diese eignen sich für das Finden von optimalen Mischverhältnissen mit der Nebenbedingung, dass die Summe aller Mischanteile 100\% ist.

Zuletzt besteht auch die Möglichkeit individuell erstellte Versuchspläne zu erstellen. In \cite{DoE_MSW} wird dies bei Versuchen im naturgetreuen Umfang umgesetzt. Die Versuche dienten dazu, die optimalen Einstellungen für eine Zerkleinerung von gemischten Siedlungsabfällen zu finden. Hier wurde der mögliche Versuchsumfang durch Faktoren beschränkt, wie die hohen Kosten der einzelnen Versuchsläufe und der Fakt dass das Inputmaterial nicht wiederverwendet werden konnte.

\todoin{Relevante Effekte werden durch FTest ermittelt}


Nachdem der passende Versuchsplan für das vorliegende Szenario aufgestellt wurde, ist es wichtig, die benötigte Anzahl an Versuchswiederholungen zu berechnen. Diese ist relevant für die Auswertung der Versuchsergebnisse.

Das liegt daran, dass die Signifikanz der Effekte mithilfe der sogenannten Varianzanalyse nach Ronald Aylmer Fisher getestet wird. Ziel dieser ist es, wie auch in \citet{Siebertz2016} beschrieben, \glqq Mittelwerte in verschiedenen Gruppen [$\dots$] auf signifikante Unterschiede zu vergleichen.\grqq Die Gruppierung richtet sich in der statistischen Versuchsplanung nach den Stufen der Faktoren.

Wie bei jedem Hypothesentest wird eine Nullhypothese aufgestellt. Man unterstellt hierbei, dass der Faktor keinen signifikanten Effekt hat \cite{Siebertz2016}.
Nun bestehen zwei Risiken für ein fehlerhaftes Entscheiden. Das sogenannte $\alpha$-Risiko beschreibt die Wahrscheinlichkeit für einen Fehler 1. Art. Das bedeutet, man lehnt die Nullhypothese ab, obwohl sie eigentlich wahr ist. In anderen Worten wird der Effekt als signifikant angenommen, obwohl er dies nicht ist.

Der zweite Mögliche Fehler ist der Fehler 2. Art. Dieser liegt vor, wenn die Nullhypothese nicht abgelehnt wird, obwohl sie falsch ist. In diesem Fall würde man also einen signifikanten Effekt nicht erkennen. Das Risiko hierfür wird $\beta$-Risiko genannt.

Wie auch in \citet{Siebertz2016} beschrieben sinken die Risiken für eine Fehlentscheidung mit zunehmend großer Stichprobengröße.

Der richtige Versuchsumfang hängt also von verschiedenen Größen ab \cite{Siebertz2016}. Zunächst ist entscheidend, wie viel Risiko für die eben genannten Fehlentscheidungen vertretbar ist. Des weiteren muss die Standardabweichung der Messgrößen miteinbezogen werden. Je höher diese Standardabweichung ist, desto mehr Versuche werden benötigt, um statistisch signifikante Effekte vom Rauschen unterscheiden zu können \cite{Siebertz2016}. Die Standardabweichung kann durch Vorversuche ermittelt werden.

Zusätzlich muss der Abstand $\delta$ festgelegt werden, ab dem Effekte als praktisch signifikant gelten \cite{Siebertz2016} sollen. Aus \todoin{Hier irgendwie auf die Tabelle hinweisen} ergibt sich die benötigte Versuchsanzahl.



\todoin{Half Normal Plot für wahre und scheinbare Effekte und Residuen}



\todoin{Anova}

\todoin{}



\section{Schüttgut im Recycling}
\subsection{Kunststoffsortierung, Baustoffrecycling, Glasrecycling, ...}
\section{Sensorbasierte Sortierung}
\section{Tablesort - Systembeschreibung}


